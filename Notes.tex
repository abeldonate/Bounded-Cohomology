\documentclass[leqno]{article}
\usepackage{verbatim}
\usepackage{array}
\usepackage{listings}
\usepackage{fancyvrb}
\usepackage{enumitem}

\usepackage[utf8]{inputenc}
\usepackage[T1]{fontenc}
\usepackage{textcomp}
\usepackage{multicol} \usepackage{mathtools}
\usepackage{amsmath}
\usepackage{wrapfig}
\usepackage{amssymb}
\usepackage{amsmath,amsfonts,amssymb,amsthm,epsfig,epstopdf,titling,url,array}
\usepackage{hyperref}
\usepackage{eso-pic}
\usepackage{pgf}
\usepackage{tikz}
\usepackage{tikz-cd}
\usepackage{graphicx}
\DeclareMathOperator{\im}{Im}

% figure support
\usepackage{import}
\usepackage{xifthen}
\pdfminorversion=7
\usepackage{pdfpages}
\usepackage{transparent}
\usepackage{xcolor}

% geometry
\usepackage{geometry}
\geometry{a4paper, margin=1in}

% paragraph length
\setlength{\parindent}{0em}
\setlength{\parskip}{1em}

\newtheorem*{theorem}{Theorem}
\newtheorem*{lemma}{Lemma}
\newtheorem*{proposition}{Proposition}
\newtheorem*{definition}{Definition}
\newtheorem*{observation}{Observation}
\newtheorem*{openproblem}{Open Problem}

\newcommand{\incfig}[1]{%
\center
\def\svgwidth{0.9\columnwidth}
\import{./figures/}{#1.pdf_tex}
}
\newcommand{\incimg}[1]{%
\center
\includegraphics[width=0.9\columnwidth]{images/#1}
}
\pdfsuppresswarningpagegroup=1

\title{Notes on Bounded Cohomology of Groups}
\author{Abel Doñate Muñoz}
\date{}

\begin{document}
\maketitle
\tableofcontents
\newpage

\section{The setting}
We briefly introduce the object we are about to study.

In a similar way we define cochain complexes in usual topological cohomology we can define in the same way cochain complexes in group cohomology

\begin{definition}[Cochain complex] Given a discrete group $\Gamma $ and an abelian group $A$, the cochain complex is defined as
\[
  C^n(\Gamma , A) = \{f: \Gamma ^{n+1}\to A\} , \qquad C^n_{b} (\Gamma , A) = \{f:\Gamma ^{n+1}\to  A | f \text{ is bounded}\}
\] 
\end{definition}

What makes a function bounded is the existence of a constant $C$ such that  $f(\gamma_0, \ldots, \gamma_n)<C \ \forall \gamma_i\in \Gamma $

\begin{definition}[Coboundary operator] This operator "raises" the degree of the cochains $\delta: C^n(\Gamma , A)\to  C^{n+1}(\Gamma , A)$
\[
  \delta f(\gamma_0,\ldots, \gamma_{n+1}) = \sum_{i=0}^{n+1} (-1)^{i}f(\gamma_0, \ldots, \hat{\gamma_i}, \ldots, \gamma_{n+1})
\] 
\end{definition}

It is straightforward to check $\delta^{k+1} \circ \delta^k = 0$

However, if we define directly the cohomology from the above cochain complex, the information encoded in the structure of the group will be lost, treating the group like a set. For that reason we introduce the $\Gamma -$ invariant cochain complex.

\begin{definition}[Cochain complex of $\Gamma -$invariants] This is the subset of the cochain complex $C^n(\Gamma , A)^\Gamma \subseteq C^n(\Gamma , A)$ whose elements fulfill
  \[
  f(\gamma_0,\ldots, \gamma_n) = f(\gamma\gamma_0,\ldots, \gamma\gamma_n) \quad \ \forall \gamma \in \Gamma 
  \] 
\end{definition}
This is what endows the cochains with the required structure to properly study a group. For convenience we will shorten the notation and we will make use of $\Gamma -$invariant cochain complex when not specified.

With this in mind we can define now the usual cohomology with the diagram in mind
\[\begin{tikzcd}
	0 & {C^0(\Gamma, A)} & {C^1(\Gamma, A)} & {C^2(\Gamma, A)} & {C^3(\Gamma, A)} & \cdots
	\arrow[from=1-1, to=1-2]
	\arrow["{\delta^0}", from=1-2, to=1-3]
	\arrow["{\delta^1}", from=1-3, to=1-4]
	\arrow["{\delta^2}", from=1-4, to=1-5]
	\arrow["{\delta^3}", from=1-5, to=1-6]
\end{tikzcd}\]

\begin{definition}[Group Cohomology] We define the group cohomology from the Cochain complex and the coboundaty operator as $\displaystyle H^n(\Gamma , A) = \frac{\ker(\delta^n)}{\im(\delta^{n-1})}$
\end{definition}

This is well defined in both cases, the bounded and the unbounded. We realize that the fact $C_b^n(\Gamma , A)\subseteq C^n(\Gamma , A)$ induces a map called comparison map
\[
c: H_b^n(\Gamma , A) \to  H^n(\Gamma , A)
\] 
The study of this comparison map is fundamental to understand how boundedness can change the setting of the problem.

\section{Inhomogeneous formulation}
To make computations in low degree we usually make use of inhomogeneous complex to transform the trivial action condition in something easier to work with.

Values of $f$ can be calculated on tuples starting with $1$.
 \[
f(\gamma_0, \ldots, \gamma_n) = f(1, \gamma_0^{-1}\gamma_1, \gamma_0^{-1}\gamma_2, \ldots, \gamma_0^{-1} \gamma_n)
\] 
and we let
\[
\begin{cases}
 g_1 = \gamma_0^{-1}\gamma_1\\
 g_2 = \gamma_1^{-1}\gamma_2\\
 \vdots\\
 g_n = \gamma_{n-1}^{-1} \gamma_n
\end{cases}
\Rightarrow \quad
f(\gamma_0, \ldots, \gamma_n) 
\leftrightarrow f(1, g_1, g_1g_2, \ldots, g_1g_2\cdots g_n) := h(g_1, g_2, \ldots, g_n)
\] 
And this defines a correspondence between the homogeneous and inhomogeneous complexes, which we denote by $\overline{C}^n$.

It is easy to check that the coboundary operator transforms in the following way 
\begin{definition}[Inhomogeneous coboundary operator]
  \[
	\overline{\delta}^nh(g_1, \ldots, g_{n+1}) = h(g_2, \ldots, g_{n+1}) + \sum_{i=1}^n (-1)^ih(g_1, \ldots, g_ig_{i+1}, \ldots, g_{n+1}) + (-1)^{n+1}h(g_1, \ldots, g_n)
  \] 
\end{definition}

Henceforth we will write $h(g_1, \ldots, g_n)$ when working with inhomogeneous complexes and $f(\gamma_0, \ldots, \gamma_n)$ with homogeneous. We forget the bar notation, since could be derived from the context.

Two classical computation follow, the groups $H^0$ and  $H^1$:

\begin{proposition}
For any discrete group $H^0(\Gamma , A) = H^0_{b} (\Gamma , A) = A$
\end{proposition}

\begin{proposition}
For any discrete group $H^1(\Gamma , A) = Hom(\Gamma , A)$  and  $H^1_{b} (\Gamma , \mathbb{Z}) = 0$
\end{proposition}

We might be surprised by the different result that boundedness gives us on the computation of $H^1_{b}$, but this is the consequence of the fact that there are no bounded homomorphisms from $\Gamma $ to $\mathbb{Z}$ or $\mathbb{R}$.

\section{Quasimorphisms}
Computing $H^2_b$ is far more complicated than  $H^0_b$ and  $H^1_b$. There is a classical result that asserts  $H^2$ is in one-to-one correspondence with the isomorphism classes of central extensions of $\Gamma $ by $A$. For the case of bounded cohomology we introduce the idea of quasimorphism.

 \begin{definition}[Quasimorphisms] Let $\Gamma $ a group. The space of quasimorphisms is defined as follows
   \[
	 QM(\Gamma ) = \{f:\Gamma \to \mathbb{R}: \ \exists C>0 \text{ such that } |f(g_1)+f(g_2)-f(g_1g_2)|<C \ \forall g_1, g_2\in \Gamma  \}   
   \] 
\end{definition}

The study of the following map is crucial for the understanding of the relationship between quasimorphisms and bounded cohomology of degree 2.
\[\begin{tikzcd}
	{QM(\Gamma)} & {\ker(H_b^2(\Gamma, \mathbb{R})} & {H^2(\Gamma, \mathbb{R}))} 
	\arrow["c", from=1-2, to=1-3]
	\arrow["A", from=1-1, to=1-2]
\end{tikzcd}\]

that sends each quasimorphism $\varphi \in QM(\Gamma )$ to $[\delta^1\varphi ]\in H_b^2$. This is trivially well defined, since $\delta^2\circ \delta^1 \varphi =0$, in such a way that $\delta^1 \varphi  \in \ker(\delta^2)$, and, thus, is mapped to zero in $H^2(\Gamma, \mathbb{R})$. Taking into account the following diagram:

\begin{minipage}{0.5\textwidth}
\[\begin{tikzcd}
	{C^1} & {C^2} & {C^3} \\
	{C^1_b} & {C^2_b} & {C^3_b}
	\arrow["{\delta^1}", from=1-1, to=1-2]
	\arrow["{\delta^2}", from=1-2, to=1-3]
	\arrow["{\delta^1_b}", from=2-1, to=2-2]
	\arrow["{\delta^1_b}", from=2-2, to=2-3]
	\arrow[hook, from=2-2, to=1-2]
	\arrow[hook, from=2-1, to=1-1]
	\arrow[hook, from=2-3, to=1-3]
\end{tikzcd}\]
\end{minipage}
\begin{minipage}{0.5\textwidth}
\[
  \Rightarrow \qquad
\begin{cases}
  \ker(\delta^2_b) \subseteq  \ker(\delta^2) \\
  \im(\delta^1_b) \subseteq \im(\delta^1)
\end{cases}
\] 
\end{minipage}

what suggest that there are some coboundaries $\delta^1 \varphi $ of $C^1$ which are not coboundaries of $C^1_b$. This is the case of unbounded $\varphi$ that leads to a quasimorphism (there are a few examples later developed such as Brook's or Rolli's quasimorphisms).

Thus, we can decompose every quasimorphism that maps to zero under the map $A$ as a sum of homomorphism and a bounded function.  $\ker(A) = B(\Gamma, \mathbb{R})\oplus Hom(\Gamma , \mathbb{R})$. By the surjectivity of the map finally we have the isomorphism
\[
\ker(A: H_b^2(\Gamma , \mathbb{R})\to H^2(\Gamma, \mathbb{R})) \cong \frac{QM(\Gamma )}{Hom(\Gamma , \mathbb{R})\oplus B(\Gamma , \mathbb{R})}
\] 

\section{The free group}
We apply now the techniques of bounded cohomology to the study of the free group. We start with the free group of two elements.

\begin{definition}[Free group on two elements] We call $F_2 = \langle a, b \rangle  $ the group of reduced words generated by the alphabet $\{a, b, a^{-1}, b^{-1}\}$
\end{definition}

Now we recall that $H^2(F, \mathbb{R}) = 0$, and the kernel of the comparison map is the whole space $H^2_b(F, \mathbb{R})$, giving the isomorphism
\[
H^2_b(F, \mathbb{R}) \cong \frac{QM(F)}{Hom(F, \mathbb{R})\oplus B(\Gamma , \mathbb{R})}
\] 
What means that we can express every nontrivial element $[\delta \varphi ]\in H^2_b$, being $\varphi $ a quasimorphism which is not bounded nor a homomorphism.

The main advantage of working with cohomology instead of homology is that cohomology is endowed with a cup product defined in the following way:

 \begin{definition}[Cup product] We define the cup product as the map
\begin{align*}
  \cup : H^n(G, \mathbb{R}) \times H^m(G, \mathbb{R}) &\to H^{n+m}(G, \mathbb{R}) \\
  ([f], [g]) & \mapsto [f]\cup [g] 
\end{align*} 
where $(f\cup g)(g_1, \ldots, g_n, g_{n+1}, \ldots, g_{n+m}) := f(g_1, \ldots, g_n)\cdot g(g_{n+1}, \ldots, g_{n+m})$
\end{definition}

Two main open questions are now natural to ask:

\begin{openproblem} Can we characterise the group $H^2_b(F, \mathbb{R})$?
\end{openproblem}

\begin{openproblem} Let  $k>0$ and  $\alpha \in H^k_b(F, \mathbb{R})$ arbitrary. Let $\varphi $ be a quasimorphism. Is  the map
  \begin{align*}
	\cup : H^2_b(F, \mathbb{R}) \times H_b^k(F, \mathbb{R}) &\to H^{k+2}_b(F, \mathbb{R}) \\
	([\delta^1\varphi], \alpha ) &\mapsto \beta 
  \end{align*}
  trivial? (i.e. $\beta =[0]$)
\end{openproblem}

\section{Known quasimorphisms} 

\subsection{Calegari Quasimorphisms}
Calegari Quasimorphisms are a generalization of small Brooks quasimorphisms. This type of quasimorphisms can be thought as a weighted sum of small Brooks quasimorphisms
\[
\varphi _\alpha := \sum_{w\in \mathcal{N}^+} \alpha _w h_w
\] 
where $\alpha $ is an alternating function $\alpha :F\to \mathbb{R}$ and the sum runs for the set of non-self-overlapping words.

\begin{definition} Let
  \[
  \kappa_\alpha (1):= \sup \left( \sum_{w\in C} |\alpha _w| \right) 
  \] 
  be the supremum over all compatible families. This is an intrinsic characteristic of the function $\alpha$.
\end{definition}

\begin{definition}[Calegari Quasimorphisms] We say $\varphi _\alpha $ is a Calegari quasimorphism if $\kappa_{\alpha }(1) < \infty$
\end{definition}

\begin{theorem} If $\varphi _\alpha \in wl^1_{Br}$, then $[\delta \varphi _\alpha ]\cup \omega =[0]$
\end{theorem}
\begin{proof}
We can proof the following statement in the same way Amontova and Bucher did it in Theorem A (b) (citation needed).

Let $\eta = \sum_{w\in \mathcal{N}^+}\eta_w$, with $\eta_w$ defined as:
\[
  \eta_w (g, h_1, \ldots, h_{k-1}) = \sum_{j=1}^{m-l+1} \chi _w (x_j\cdot \ldots\cdot x_{j+l-1})\omega (z_{j+l}(g), h_1, \ldots, h_{k-1})
\] 
Now we let $\beta = \varphi_\alpha \cup \omega +\delta \eta = \sum_{w\in \mathcal{N}^+}\left( \alpha _w h_w \cup \omega +\delta\eta_w \right) = \sum_{w\in \mathcal{N}^+}$. One can check that $\delta \beta = (\delta \varphi _\alpha )\cup \omega $, and we only need to show that $\beta $ is bounded.

We know from Amontova and Bucher that  $\|\beta _w\| \le (|w|-1)\|\omega \|_{\infty}$, so $\|\beta _w\|\le \sum_{w\in \mathcal{N}^+}((|w|-1)\alpha _w)\|\omega \|_{\infty}<\infty$, because $\varphi _\alpha \in wl^1_{Br}$
\end{proof}




\end{document}
