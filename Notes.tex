\documentclass[leqno]{article}
\usepackage{verbatim}
\usepackage{array}
\usepackage{listings}
\usepackage{fancyvrb}
\usepackage{enumitem}

\usepackage[utf8]{inputenc}
\usepackage[T1]{fontenc}
\usepackage{textcomp}
\usepackage{multicol} \usepackage{mathtools}
\usepackage{amsmath}
\usepackage{wrapfig}
\usepackage{amssymb}
\usepackage{amsmath,amsfonts,amssymb,amsthm,epsfig,epstopdf,titling,url,array}
\usepackage{hyperref}
\usepackage{eso-pic}
\usepackage{pgf}
\usepackage{tikz}
\usepackage{tikz-cd}
\usepackage{graphicx}
\DeclareMathOperator{\Ima}{Im}

% figure support
\usepackage{import}
\usepackage{xifthen}
\pdfminorversion=7
\usepackage{pdfpages}
\usepackage{transparent}
\usepackage{xcolor}

% geometry
\usepackage{geometry}
\geometry{a4paper, margin=1in}

% paragraph length
\setlength{\parindent}{0em}
\setlength{\parskip}{1em}

\newtheorem*{theorem}{Theorem}
\newtheorem*{lemma}{Lemma}
\newtheorem*{proposition}{Proposition}
\newtheorem*{definition}{Definition}
\newtheorem*{observation}{Observation}

\newcommand{\incfig}[1]{%
\center
\def\svgwidth{0.9\columnwidth}
\import{./figures/}{#1.pdf_tex}
}
\newcommand{\incimg}[1]{%
\center
\includegraphics[width=0.9\columnwidth]{images/#1}
}
\pdfsuppresswarningpagegroup=1

\title{Notes on Bounded Cohomology of Groups}
\author{Abel Doñate Muñoz}
\date{}

\begin{document}
\maketitle
\tableofcontents
\newpage

\section{The setting}
We briefly introduce the object we are about to study.

In a similar way we define cochain complexes in usual topological cohomology we can define in the same way cochain complexes in group cohomology

\begin{definition}[Cochain complex] Given a discrete group $\Gamma $ and an abelian group $A$, the cochain complex is defined as
\[
  C^n(\Gamma , A) = \{f: \Gamma ^{n+1}\to A\} , \qquad C^n_{b} (\Gamma , A) = \{f:\Gamma ^{n+1}\to  A | f \text{ is bounded}\}
\] 
\end{definition}

\begin{definition}[Coboundary operator] This operator "raises" the degree of the cochain complex $\delta: C^n(\Gamma , A)\to  C^{n+1}(\Gamma , A)$
\[
  \delta f(\gamma_0,\ldots, \gamma_{n+1}) = \sum_{i=0}^{n+1} (-1)^{i}f(\gamma_0, \ldots, \hat{\gamma_i}, \ldots, \gamma_{n+1})
\] 
\end{definition}

It is straightforward to check $\delta^{k+1} \circ \delta^k = 0$

However, if we define directly the cohomology from the above cochain complex, the information encoded in the structure of the group will be lost, treating the group like a set. For that reason we introduce the $\Gamma -$ invariant cochain complex.

\begin{definition}[$\Gamma -$invariant cochain complex] This is the subset of the cochain complex $C^n(\Gamma , A)^\Gamma \subseteq C^n(\Gamma , A)$ whose elements fulfill
  \[
  f(\gamma_0,\ldots, \gamma_n) = f(\gamma\gamma_0,\ldots, \gamma\gamma_n) \quad \ \forall \gamma \in \Gamma 
  \] 
\end{definition}
This is what endows the chains the required structure to properly study a group. For convinience we will shorten the notation and we will make use always $\Gamma -$invariant cochain complex when not specified.

With this in mind we can define now the usual cohomology with the diagram in mind
\[\begin{tikzcd}
	0 & {C^0(\Gamma, A)} & {C^1(\Gamma, A)} & {C^2(\Gamma, A)} & {C^2(\Gamma, A)} & \cdots
	\arrow[from=1-1, to=1-2]
	\arrow["{\delta^0}", from=1-2, to=1-3]
	\arrow["{\delta^1}", from=1-3, to=1-4]
	\arrow["{\delta^2}", from=1-4, to=1-5]
	\arrow["{\delta^3}", from=1-5, to=1-6]
\end{tikzcd}\]

\begin{definition}[Cohomology] We define the group cohomology from the Cochain complex and the coboundaty operator as $\displaystyle H^n = \frac{\ker(\delta^n)}{\Ima(\delta^{n-1})}$
\end{definition}

This is well defined in both cases, the bounded and the unbounded. We realize that the fact $C_b^n(\Gamma , A)\subseteq C^n(\Gamma , A)$ induces a map called comparison map
\[
c: H_b^n(\Gamma , A) \to  H^n(\Gamma , A)
\] 
The study of this comparison map is fundamental to understand how boundedness can change the setting of the problem.

\section{Inhomogeneous formulation}
To make computations in low degree we usually make use of inhomogeneous complex to transform the trivial action condition in something more "computable".


\end{document}
